% !Mode:: "TeX:UTF-8"
\documentclass{article}
\usepackage{CJK}


\usepackage{tabu}


\usepackage{graphicx}


%\usepackage{makecell}

\usepackage{array}


\begin{document}
	\begin{CJK}{UTF8}{gbsn}
		
		下面这个是正常的表格. \\在列设定为center的时候,是没办法控制列宽的,除非用下面的复杂的表格,有一个简单的办法:直接在列的前后内容(比如列头的前后,插入一些空格\quad,就可以增加宽度了)
		\begin{table}[h!]
			\caption{Results of multi-label CRF model with different feature sets.}
			\begin{tabular}{c c c c  }
				\hline
				\rule{0pt}{8pt}\textbf{Feature set} & \textbf{Precision} & \textbf{Recall} & \textbf{F${_1}$-Score} \tabularnewline
				\hline
				\rule{0pt}{8pt}CRF + BOW & 0.8189 & 0.5795 & 0.6787\tabularnewline
				%\hline
				\rule{0pt}{8pt}CRF + BOW + POS & 0.8052 & 0.6086 & 0.6932\tabularnewline
				\hline
			\end{tabular}
		\end{table}
		
		
		
		
		
		下面要展示如何制作复杂的表格
		
		法一:使用tabu宏包
		
		\tabulinesep =2ex
		\begin{tabu} to 10cm {|X[$$ c] |X[$$ c]|X[$$4 c]|}
			\hline  s & b &  f(x_{n})=\frac{1}{\sqrt{K_{0}}}\int_{-\infty}^{\infty}F(k){\rm{e}}^{\pm{\rm{i}}kx_{n}}{\rm{d}}k\\
			\hline T & -1 & 2\\
			\hline
			\end{tabu}
		
		
		
		
		法二:使用graphicx宏包. 这个方法和makecell宏包冲突
		
		
		\begin{tabular}{| c | >{\centering}c | >{\centering\vspace{6mm}}m{6cm}<{\vspace{6mm}} |}
		\hline
		\scalebox{1}[1]{s} & \scalebox{1}[1]{\setlength{\fboxsep}{0pt}\fbox{b}} &  ${\displaystyle f(x_{n})=\frac{1}{\sqrt{K_{0}}}\int_{-\infty}^{\infty}F(k)e^{\pm ikx_{n}}\,{\rm d}k}$ \tabularnewline\hline
		T &1&2\tabularnewline\hline
		\end{tabular}
		
		\begin{verbatim}
		说明:
		
		垂直居中效果和scalebox无关。
		
		重点在于:
		
		>{\centering\vspace{6mm}}m{6cm}<{\vspace{6mm}}
		tabu 倒是倒是不错的包,不过我实际上是用 lyx 写的东西,不方便用 tabu
		
		最后还是把 arraystretch 调小了,看上去差不多就得了
		
		m{6cm}表示列的宽度,两个vspace表示表格内容上下的空白区高度
		
		\begin{verbatim}
		想要不居中,则去掉\centering即可:>{\vspace{6mm}}m{6cm}<{\vspace{6mm}}
		
		(发现有的时候调试centering和vspace之类的时候会报错,可能原因是换行要用\tabularnewline,不要用\\)
		\end{verbatim}
		
		
		
		法三:makecell宏包
		\iffalse
		\makegapedcells
		\setcellgapes{3pt}
		\newsavebox{\mybox}
		\newcolumntype{X}[1]{>{\begin{lrbox}{\mybox}}c<{\end{lrbox}\makecell[#1]{\fbox{\usebox\mybox}}}}
		\begin{tabular}{|X{cc}|X{cc}|X{cc}|}
		\hline
		A \rule{3mm}{15mm}  &  B \rule{3mm}{18mm} & \rule{3mm}{10mm} ${\displaystyle f(x_{n})=\frac{1}{\sqrt{K_{0}}}\int_{-\infty}^{\infty}F(k)e^{\pm ikx_{n}}\,{\rm d}k}$ \tabularnewline\hline
		T &1sssssssss&2\\\hline
		\end{tabular}
		
		\begin{verbatim}
		说明:
		重新定义列格式 X{..},比如 X{cc} 就是水平和竖直都是居中对齐
		
		\newsavebox{\mybox}
		\newcolumntype{X}[1]{%
			>{\begin{lrbox}{\mybox}}%
			c%
			<{\end{lrbox}\makecell[#1]{\usebox\mybox}}%
		}
		\end{verbatim}
		\fi
		
		法三补充实例1:
		
		\begin{table}[h!]
		\caption{Results of multi-label CRF model with different feature sets.}
		\begin{tabular}{p{8cm} c c c  }
		\hline
		\rule{0pt}{8pt}\makebox[9cm][c]{\textbf{Feature set}} & \textbf{Precision} & \textbf{Recall} & \textbf{F${_1}$-Score} \tabularnewline
		\hline
		\rule{0pt}{8pt}CRF + BOW & 0.8189 & 0.5795 & 0.6787\tabularnewline
		%\hline
		\rule{0pt}{8pt}CRF + BOW + POS & 0.8052 & 0.6086 & 0.6932\tabularnewline
		%\hline
		\rule{0pt}{8pt}CRF + BOW + POS + capitalization & 0.8169 & 0.6299 & 0.7113\tabularnewline
		%\hline
		\rule{0pt}{8pt}CRF + BOW + POS + capitalization + case pattern& 0.8148 & 0.6364 & 0.7146\tabularnewline
		%\hline
		\rule{0pt}{8pt}CRF + BOW + POS + capitalization + case pattern + word representation& 0.8287 & 0.6872 & 0.7514\tabularnewline
		\hline
		\end{tabular}
		\end{table}
		
		
		
		
		
		\begin{table}[h!]
		\caption{Results of multi-label CRF model with different feature sets.}
		\begin{tabular}{>{\raggedright\vspace{0mm}} m{7cm}<{\vspace{0mm}} c c c  }
		\hline
		\rule{0pt}{8pt}\makebox[7cm][c]{\textbf{Feature set}} & \textbf{Precision} & \textbf{Recall} & \textbf{F${_1}$-Score} \\
		\hline
		\rule{0pt}{8pt}CRF + BOW & 0.8189 & 0.5795 & 0.6787\\
		%\hline
		\rule{0pt}{8pt}CRF + BOW + POS & 0.8052 & 0.6086 & 0.6932\\
		%\hline
		\rule{0pt}{8pt}CRF + BOW + POS + capitalization & 0.8169 & 0.6299 & 0.7113\\
		%\hline
		\rule{0pt}{8pt}CRF + BOW + POS + capitalization + case pattern& 0.8148 & 0.6364 & 0.7146\\
		%\hline
		\rule{0pt}{8pt}CRF + BOW + POS + capitalization + case pattern + word representation& 0.8287 & 0.6872 & 0.7514\\
		\hline
		\end{tabular}
		\end{table}
		
		\begin{verbatim}
		(m{7cm}<{\vspace{0mm}}所在的那列下面可以加“\\”表格内换行)
		\end{verbatim}
		
	\end{CJK}
\end{document}